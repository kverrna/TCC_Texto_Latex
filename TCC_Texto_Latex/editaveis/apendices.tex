
\begin{apendicesenv}

\partapendices

\chapter{APOSENTADORIA NO BRASIL}

Aposentar-se é o ato de afastar-se de atividades profissionais. Após um período de contribuição em algum plano de aposentadoria, seja ele público ou privado, pode se ter um afastamento remunerado por tempo indeterminado. No Brasil, a aposentadoria é um direito garantido pela Constituição Federativa do Brasil (CF), no artigo 201 (BRASIL,1988). Tem-se para tanto um Ministério responsável por amparar o trabalhador quanto aos seus direitos. Existe hoje no Brasil três Regimes Previdenciários: (i) o regime próprio (BRASIL, Constituição,1988, artigo 40), este para servidores públicos, no qual o órgão em que ele está lotado possui um regime previdenciário; (ii) o regime geral (BRASIL, Constituição,1988, artigo 201), este de natureza obrigatória e vínculo automático ao trabalhador de empresas privadas e públicas, sempre quando essas não possuírem um regime previdenciário próprio; (iii) e o regime complementar (BRASIL, Constituição,1988,artigo 202) (BRASIL, Emenda Constitucional nº 20,1998), este de caráter facultativo. Pode ser do tipo fechado (BRASIL, Lei Complementar 108, 2001), ou seja, somente para trabalhadores que possuem vínculos empregatícios em uma determinada empresa. Mas, também pode ser do tipo aberto (BRASIL, Lei Complementar 109, 2001), ou seja, não vinculado ao trabalho, e normalmente gerido por empresas financeiras.

O regime geral de previdência é administrado pelo Instituto Nacional do Seguro Nacional, INSS. Trata-se de um Órgão vinculado ao Ministério da Previdência Social, criado em 24 de junho de 1990 (BRASIL, Decreto nº 99.350, 1990). Entretanto, o direito previdenciário existe desde 24 de janeiro de 1923, com o Decreto 4.682/23 (BRASIL, 1923). Os decretos supracitados garantem o direito à previdência ao trabalhador, denominado assegurado, e seus dependentes. São reconhecidos como dependentes (BRASIL,Lei nº 8.213, 1991, Art. 16): (i) o cônjuge, a companheira, o companheiro e o filho não emancipado, de qualquer condição, menor de 21 anos ou inválido ou que tenha deficiência intelectual que o torne absoluta ou relativamente incapaz, assim declarado judicialmente; (ii) os pais; (iii) e o irmão não emancipado, de qualquer condição, menor de 21 anos ou que tenha deficiência intelectual. No caso destes últimos, o INSS dispõe de pensão por morte e auxilio reclusão, quando o assegurado vinculado morre ou mesmo cumpre pena em regime carcerário.

Vale ressaltar ainda que todas as pessoas físicas brasileiras são asseguradas, as quais são divididas em segurados obrigatórios e facultativos (BRASIL, Lei nº 8.213, 1991 Art. 11). São denominados segurados obrigatórios: (i) todos os trabalhadores que prestam serviços de natureza urbana ou rural à empresa, em caráter não eventual, sob sua subordinação e mediante remuneração, inclusive como diretor empregado; (ii) empregados domésticos, aqueles que prestam serviços de natureza contínua a uma pessoa ou família, no âmbito residencial desta, em atividades sem fins lucrativos; (iii) contribuinte individual, que são fazendeiros, pescadores e garimpeiros; (iv) trabalhadores avulsos, aqueles que prestam serviços para diversas empresas, sem vínculo empregatício, serviço de natureza urbana ou rural. São denominados segurados facultativos, todos os indivíduos maiores de 14 anos que se afiliam ao regime geral de previdência social, mediante contribuição, desde que não incluído nas disposições do artigo 11 (BRASIL, Lei nº8.213, 1991, Art. 13).

Os segurados obrigatórios são vinculados de maneira automática ao INSS. Para estes são oferecidos oito benefícios (BRASIL, Lei nº 8.213,1991, Art. 18), expostos abaixo:

\begin{itemize}
\item Aposentadorias

		\begin{itemize}
		\item Aposentadoria por idade: 65 para homens e 60 para mulheres;
		\item Aposentadoria por tempo de contribuição: 35 para homes e 30 mulheres;
		\item Aposentadoria por invalidez: em virtude de acidentes que invalide o segurado, de maneira permanente, à executar atividades profissionais;
		\item Aposentadoria especial: para profissionais que lidam com agentes químicos ou biológicos, os quais são considerados nocivos à sua saúde;

		\end{itemize}
\item Auxílios

		\begin{itemize}
		\item Auxílio doença: em períodos de doença, a qual impeça o segurado na realização de atividades profissionais de maneira temporária;
		\item Auxilio acidente: para acidentes ligados ao trabalho, que embora não tenha sido a causa única, tenha contribuído para morte do segurado, para redução ou perda da sua capacidade para o trabalho, ou produzido lesão que exija atenção médica para sua recuperação.

		\end{itemize}
\item Salários

		\begin{itemize}
		\item Salário maternidade: pago durante 120 dias, com início no período de 28 dias antes do parto e a data de ocorrência deste, observadas as situações e condições previstas na legislação no que concerne à proteção à maternidade;
		\item Salário família: pagos mensalmente ao segurado empregado, exceto ao doméstico, e ao segurado trabalhador avulso, na proporção do respectivo número de filhos ou equiparados nos termos do  \S 2º  da lei 8.213/91 (BRASIL, 1991).
		\end{itemize}
\end{itemize}

Vale ressaltar que ao segurado que optar pela aposentadoria por tempo de contribuição, por exemplo, um homem que iniciou seu período de contribuição aos 18 anos e completou seus 35 anos de contribuição aos 53 anos, este terá seu valor de benefício corrigido pelo fator previdenciário (BRASIL, Lei Nº 9.876, 1999). Nesse caso, calcula-se a média aritmética simples dos maiores salários-de-contribuição correspondentes a oitenta por cento de todo o período contributivo, multiplicada pelo fator previdenciário (BRASIL, Lei Nº 9.876, 1999). A fórmula do fator previdenciário é exposta na Figura seguinte.

\textbf{Figura 2: Fator Previdenciário}

Onde Tc representa o tempo de contribuição; A corresponde à alíquota, que é um valor estipulado pelo Ministério da Previdência Social (MINISTÉRIO DA PREVIDÊNCIA SOCIAL, 2014); Es é a expectativa de sobrevida, calculada pelo IBGE, e Id é a idade do segurado. Desta forma, o segurado que possui tempo de contribuição suficiente para requerer uma aposentadoria por tempo de contribuição não receberá o valor integral do benefício. O fator previdenciário é de uso obrigatório na aposentadoria por tempo de contribuição e facultativo na aposentadoria por idade.

Atualmente, o teto máximo oferecido pelo INSS é de R\$ 4.390,24 (MINISTÉRIO DA PREVIDÊNCIA SOCIAL, 2014), valor este reajustado anualmente. Ao assegurado que tem remuneração de valor até o teto máximo, a aposentadoria garantida por lei é suficiente para a manutenção do seu padrão de vida. Porém, ao segurado obrigatório, que possui remuneração superior ao teto máximo oferecido, o valor que será recebido a título de aposentadoria não será o suficiente para manter seus padrões. Estes últimos deverão procurar outros meios para que seja possível manter o seu nível de renda, sejam previdências privadas oferecidas por instituições financeiras ou investimentos feitos no mercado de capitais brasileiro e/ou estrangeiros. Se optar por realizar investimentos no mercado de capitais, este pode ser feito através de fundos de investimentos ou mesmo pessoalmente, apoiado por especialistas no assunto bem como utilizando ferramentas adequadas. Algumas dessas ferramentas são descritas na próxima subseção. O principal tipo de investidor que se pretende auxiliar com os estudos e suporte tecnológico providos nesse Trabalho de Conclusão de Curso é o investidor descrito neste parágrafo.

\section*{REFERÊNCIAS}

BRASIL. Constituição (1988). Constituição da República Federativa do Brasil. Brasília-DF,Senado Federal,2014.

BRASIL, Decreto nº 99.350. Cria o Instituo Nacional do Seguro Nacional (INSS) define sua estrutura básica e o Quadro Distributivo de Cargos e Funções do Grupo-Direção e Assessoramento Superiores de suas Unidades Centrais e dá outras providências. Presidência da Republica. Brasília-DF ,1990. Disponível em < http://www.planalto.gov.br/ccivil\_03/decreto/Antigos/D99350.htm> acesso em 26 de maio de 2014.

BRASIL. Decreto 4.682 . Lei que cria, em cada uma das empresas de estradas de ferro existentes no país, uma caixa de aposentadoria e pensões para os respectivos empregados. Presidência da República. Rio de Janeiro-RJ, 1923. Disponível em <http://www.planalto.gov.br/ccivil\_03/decreto/Historicos/DPL/DPL4682.htm > Acesso em 26 de maio de 2014.

BRASIL, Lei da dos Benefícios da Previdência Social. Dispõe sobre a contribuição previdenciária do contribuinte individual, o cálculo do benefício, altera dispositivos das Leis nos 8.212 e 8.213, ambas de 24 de julho de 1991, e dá outras providências. Lei nº 9.876/91. Presidência da República, Casa Civil. Brasília-DF, 1991. Disponível em < http://www.planalto.gov.br/ccivil\_03/leis/l8213cons.htm> acesso em 25 de maio de 2014.

BRASIL. Lei Complementar nº 108. Dispõe sobre a relação entre a União, os Estados, o Distrito Federal e os Municípios, suas autarquias, fundações, sociedades de economia mista e outras entidades públicas e suas respectivas entidades fechadas de previdência complementar, e dá outras providências. Presidência da Republica. Brasília-DF, 2001. Disponível em  <http://www.planalto.gov.br/ccivil\_03/leis/lcp/lcp108.htm> Acesso em 26 de maio de 2014.

BRASIL. Lei Complementar nº 109. Dispõe sobre o Regime de Previdência Complementar e dá outras providências. Presidência da Republica. Brasília-DF, 2001. Disponível em <http://www.planalto.gov.br/ccivil\_03/leis/lcp/lcp109.htm> Acesso em 26 de maio de 2014.

MINISTERIO DA PREVIDENCIA SOCIAL. Fator Previdenciário. Brasília –DF. Disponível em <http://www.previdencia.gov.br/informaes-2/fator-previdencirio-2/> acesso em 25 de maio de 2014.

MINISTÉRIO DA PREVIDENCIA SOCIAL. Portaria Interministerial MPS/MF Nº 19, de 10 de Janeiro de 2014-DOU de 13/01/2014. Brasília-DF. Disponível em <http://www3.dataprev.gov.br/sislex/paginas/65/mf-mps/2014/19.htm>   acesso em 25 de maio de 2014.

\chapter{ALGORITMO DARK CLOUD EM MQL4}
\lstinputlisting[language=C,frame=single]{sourceCode/codeRamon.c}

\chapter{ALGORITMO DARK CLOUD EM JAVA+ JADE}
\section{Classe DarkCloud}
\lstinputlisting[language=Java,frame=single]{sourceCode/darkCloud.java}

\section{Classe abstrata Pattern}
\lstinputlisting[language=Java,frame=single]{sourceCode/pattern.java}

\section{Classe CandleStick}
\lstinputlisting[language=Java,frame=single]{sourceCode/candleStick.java}


\section{Classe LoadCandleStickForCSVFile}
\lstinputlisting[language=Java,frame=single]{sourceCode/LoadCandleStickForCSVFile.java}

\section{Agente experimentalGrafista}
\lstinputlisting[language=Java,frame=single]{sourceCode/Grafista.java}

\chapter{ALGORITMO DARK CLOUD EM PYTHON}
\section{Bloco DarkCloud}
\lstinputlisting[language=Python,frame=single]{sourceCode/darkCloud.py}

\section{Bloco Bloco FindPatternDarkCloud}
\lstinputlisting[language=Python,frame=single]{sourceCode/FindPatternDarkCloud.py}

\chapter{RELATÓRIO PESQUISA - AÇÃO 001: COLETA DE IMPRESSÕES COM USUÁRIOS}
\section{INTRODUÇÃO}
Esta coleta de impressões com potenciais usuários teve como principal objetivo verificar  o objetivo especifico proposto: “Abstrair a complexidade de cálculos financeiros comumente utilizados na Análise Técnica. Assim, essa complexidade não é sentida pelo usuário, deixando a mesma a cargo da ferramenta”. Vale ressaltar que essa avaliação teve como foco a interface gráfica da ferramenta com os usuários, onde cada usuário recebeu um formulário com questões objetivas e um campo livre para sugestões.
A Ferramenta foi apresentada aos potenciais usuários em modo de simulação no período de 1 de março de 2013 a 1 de março de 2014, a versão do JADE utilizado é a 4.3.3 de 10 de dezembro de 2014, a versão do Grais era a 2.4.3 e a versão do java era a 1.7.0\_79. A ferramenta foi simulada ainda em um notebook da marca Apple modelo MacBook Pro (Retina, 13-inch, Late 2013) com processador 2.6 GHz Intel Core i5, com 8 GB 1600 MHz DDR3 de Memória RAM e com o Sistema operacional OS X Yosemite versão 10.10.3

\section{PARTICIPANTES}
A avaliação ocorreu em maio de 2014 e participaram da avaliação 6 participante onde: 1 classifica-se no perfil Corajoso; 1 classifica-se no perfil Moderado e 4 classificam-se no perfil Conservador. Foi entregue a cada participante um formulário não identificado com 7 questões objetivas e uma questão aberta. 

\section{IMPRESSÕES COLETADAS}
Quando questionados sobre as cores adotadas na ferramenta, 5 pessoas consideraram boas e 1 pessoa considerou ótimas. Quando questionados sobre a facilidade de compreensão da ferramenta, 3 pessoas consideraram como bom, 1 pessoa considerou razoável, 1 pessoa considerou como ruim e 1 se absteve. Quando solicitados observações gerais para ferramenta destacaram-se: (i) feedback indicando o carregamento e o processo da ferramenta; (ii) atualizar página principal automaticamente; (iii) dispor de um canal de dúvidas; (iv) não adotar estratégias baseadas em Médias Móveis; (v) utilizar javascript na interface; (vi) Executar a ferramenta em duas máquinas; (vii) Informações sobre estratégias e uso da ferramenta.

As sugestões apresentadas pelos usuários geram demanda por melhorias na interface gráfica, como esperado, e uma demanda na melhoria nas estratégias adotadas, esta demanda coincidiu com os resultados obtidos no Relatório 02, onde verificou-se que o uso de estratégias baseadas em médias móveis para o perfil Conservador se mostrou ineficiente. Para atender a demanda de melhorias na interface gráfica foi necessário realizar pequenas capacitações em tecnologias front-end, tais como javaScript e jQuery. Para atender a demanda de melhorias nas estratégias, foi necessário criar novas estratégias baseadas em outros indicadores financeiros e simular novamente a estratégia.

\section{AÇÕES TOMADAS} 
Foi criado um mecanismos de onde o usuário tem a possibilidade de enviar dúvidas, sugestões, reclamações e elogios. figura X. Foi criada uma página com informações sobre a ferramenta bem como suas estratégias, figura X. Foi adicionado no projeto grails o plugin JQuery UI v.1.10.4 e jQuery v.1.11.1. No tocante a estratégias foi implementado uma nova estratégias baseada no indicador financeiro Bandas de Bollinger.


\section{CONCLUSÃO} 

Foram identificadas nesta coleta de impressões, demandas por melhorias onde para atender foi necessário buscar tecnologias front-end que auxiliassem a implementação. Foi implementado ainda um mecanismo por onde o usuário tem a possibilidade de enviar dúvidas, sugestões, reclamações e elogios. Foi criado ainda uma página com informações sobre as estratégias adotadas pela ferramenta. Por fim, pode-se concluir que esta coleta de impressões obteve sucesso. 

\chapter{RELATÓRIO PESQUISA - AÇÃO 002: COLETA DE IMPRESSÕES ATRAVÉS DE SIMULAÇÃO DE ESTRATÉGIA POR PERFIL DE USUÁRIO}
\section{INTRODUÇÃO}

Esta coleta de impressões com potenciais usuários teve como principal objetivo verificar  o objetivo especifico proposto: “Abstrair a complexidade de cálculos financeiros comumente utilizados na Análise Técnica. Assim, essa complexidade não é sentida pelo usuário, deixando a mesma a cargo da ferramenta”. Vale ressaltar que essa avaliação teve como foco as estratégias adotadas na ferramenta.

As estratégias foram simuladas no período no período de 1 de janeiro de 2012 a 1 de março de 2015, a versão do JADE utilizado é a 4.3.3 de 10 de dezembro de 2014, a versão do Grais era a 2.4.3 e a versão do java era a 1.7.0\_79. A ferramenta foi simulada ainda em um notebook da marca Apple modelo MacBook Pro (Retina, 13-inch, Late 2013) com processador 2.6 GHz Intel Core i5, com 8 GB 1600 MHz DDR3 de Memória RAM e com o Sistema operacional OS X Yosemite versão 10.10.3.

\section{IMPRESSÕES COLETADAS}

Todas estratégias de todos perfis foram simuladas de maneira conjunta de acordo com o perfil relacionado, as tabelas apresentadas abaixo contém os dados detalhados das simulações.

\textbf{Colocar Tabelas aqui }

\section{TABELAS}
\subsection{Dados Simulação Perfil Conservador}

\begin{center}
\begin{longtable}{| p{2cm} | p{10cm} |p{2cm} |}
\caption*{Estratégias Perfil Conservador e Resultados} \\
\hline
\textbf{Agente} & \textbf{stratégia} & \textbf{Percentual de Ganho} \\ \hline
\endfirsthead
\multicolumn{2}{c}%
{\tablename\ \thetable\ -- \textit{Continuação da página anterior}} \\
\hline
\textbf{Agente} & \textbf{stratégia} & \textbf{Percentual de Ganho} \\ \hline
\endhead
\hline \multicolumn{2}{c}{\textit{Continuaçao na próxima página}} \\
\endfoot
\hline
\endlastfoot

	A1 & Média Móvel Simples 13/21 & -29,06\% \\ \hline
	A2 & Media Móvel Simples 21/34 & -31,74\% \\ \hline
	A3 & Media Móvel Exponencial 21/34 & 34,96\% \\ \hline
	A4 & Media Móvel Simples  13/21 & 4,93\% \\ \hline
	A5 & Media Móvel Simples 21/34. & -60,41\% \\ \hline
	A6 & Media Móvel Exponencial 21/34. & 8,26\% \\ \hline
	A7 & Media Móvel Simples  13/21. & 59,91\% \\ \hline
	{} & \textbf{TOTAL GERAL} & \textbf{7,8\%} 
	
\label{t09}
\end{longtable}
\end{center} 

\begin{center}
\begin{longtable}{| p{4cm} | p{4cm} |p{4cm} |p{2cm} |}
\caption*{Agente A1: Ação GEPA3.SA} \\
\hline
\textbf{DATA} & \textbf{Ordem} & \textbf{R\$} & \textbf{Ganho (\%)}\\ \hline
\endfirsthead
\multicolumn{2}{c}%
{\tablename\ \thetable\ -- \textit{Continuação da página anterior}} \\
\hline
\textbf{DATA} & \textbf{Ordem} & \textbf{R\$} & \textbf{Ganho (\%)}\\ \hline
\endhead
\hline \multicolumn{2}{c}{\textit{Continuaçao na próxima página}} \\
\endfoot
\hline
\endlastfoot
	1/5/2013	&Buy	&64.02	&{}\\ \hline
	3/0/2014	&Buy	&62.04	&{}\\ \hline
	1/2/2014	&Sell	&47.51	&-26.00\%\\ \hline
	1/4/2014	&Buy	&48		&{}\\ \hline
	{} 		&{}		&\textbf{\% Acumulado} 	&\textbf{-26,00\%}

\label{t1}
\end{longtable}
\end{center}

\begin{center}
\begin{longtable}{| p{4cm} | p{4cm} |p{4cm} |p{2cm} |}
\caption*{Agente A1: Ação SHOW3.SA} \\
\hline
\textbf{DATA} & \textbf{Ordem} & \textbf{R\$} & \textbf{Ganho (\%)}\\ \hline
\endfirsthead
\multicolumn{2}{c}%
{\tablename\ \thetable\ -- \textit{Continuação da página anterior}} \\
\hline
\textbf{DATA} & \textbf{Ordem} & \textbf{R\$} & \textbf{Ganho (\%)}\\ \hline
\endhead
\hline \multicolumn{2}{c}{\textit{Continuaçao na próxima página}} \\
\endfoot
\hline
\endlastfoot
	
	5/1/2012	&Buy	&13.2	&{}\\ \hline
	4/4/2012	&Sell	&15.5	&17.42\%\\ \hline
	5/6/2012	&Buy	&17.37	&{}\\ \hline
	2/7/2012	&Sell	&17.3	&-0.40\%\\ \hline
	5/7/2012	&Buy	&17.39	&{}\\ \hline
	3/8/2012	&Sell	&14.1	&-18.92\%\\ \hline
	1/0/2013	&Buy	&7.62	&{}\\ \hline
	5/1/2013	&Sell	&7.68	&0.79\%\\ \hline
	3/2/2013	&Buy	&7.9	&{}\\ \hline
	4/4/2013	&Sell	&8.44	&6.84\%\\ \hline
	2/4/2013	&Buy	&9.19	&{}\\ \hline
	4/5/2013	&Sell	&6.86	&-25.35\%\\ \hline
	2/7/2013	&Buy	&8.59	&{}\\ \hline
	3/8/2013	&Sell	&7.7	&-10.36\%\\ \hline
	4/11/2013	&Buy	&5.35	&{}\\ \hline
	4/1/2014	&Sell	&4.8	&-10.28\%\\ \hline
	4/3/2014	&Buy	&4.93	&{}\\ \hline
	2/4/2014	&Sell	&5		&1.42\%\\ \hline
	4/4/2014	&Buy	&4.88	& \\ \hline
	2/5/2014	&Sell	&5.2	&6.56\%\\ \hline
	2/6/2014	&Buy	&5.33	&{}\\ \hline
	4/7/2014	&Sell	&5.1	&-4.32\%\\ \hline
	4/11/2014	&Buy	&2.88	&{}\\ \hline
	5/0/2015	&Sell	&2.98	&3.47\%\\ \hline

	{} 		&{}		&\textbf{\% Acumulado} 	&\textbf{-26,00\%}

\label{t3}
\end{longtable}
\end{center}


\begin{center}
\begin{longtable}{| p{4cm} | p{4cm} |p{4cm} |p{2cm} |}
\caption*{Agente A1: Ação TELB4.SA} \\
\hline
\textbf{DATA} & \textbf{Ordem} & \textbf{R\$} & \textbf{Ganho (\%)}\\ \hline
\endfirsthead
\multicolumn{2}{c}%
{\tablename\ \thetable\ -- \textit{Continuação da página anterior}} \\
\hline
\textbf{DATA} & \textbf{Ordem} & \textbf{R\$} & \textbf{Ganho (\%)}\\ \hline
\endhead
\hline \multicolumn{2}{c}{\textit{Continuaçao na próxima página}} \\
\endfoot
\hline
\endlastfoot
	
		5/1/2012	&Buy		&14.79	&\\\hline
		4/3/2012	&Sell	&13.69	&-7.44\%\\\hline
		1/9/2012	&Buy		&7.37	& \\\hline
		3/11/2012	&Sell	&6.61	&-10.31\% \\\hline
		4/11/2012	&Buy		&8.04	& \\\hline
		1/0/2013	&Sell	&7.09	&-11.82\%\\\hline
		5/9/2013	&Buy		&2.87 &	\\\hline
		4/9/2013	&Sell	&2.7		&-5.92\%\\\hline
		5/10/2013	&Buy		&2.82	&\\\hline
		4/10/2013	&Sell	&2.84	&0.71\%\\\hline
		5/3/2014	&Buy		&2.52	&\\\hline
		5/4/2014	&Sell	&2.37	&-5.95\%\\\hline
		5/5/2014	&Buy		&2.49	&\\\hline
		4/6/2014	&Sell	&2.33	&-6.43\%\\\hline
		2/8/2014	&Buy		&2.26	&\\\hline
		4/9/2014	&Sell	&1.87	&-17.26\%\\\hline
		5/2/2015	&Buy		&1.13	&\\\hline

	{} 		&{}		&\textbf{\% Acumulado} 	&\textbf{27,3\%}

\label{t1}
\end{longtable}
\end{center}


\begin{center}
\begin{longtable}{| p{4cm} | p{4cm} |p{4cm} |p{2cm} |}
\caption*{Agente A2: Ação TERI3.SA} \\
\hline
\textbf{DATA} & \textbf{Ordem} & \textbf{R\$} & \textbf{Ganho (\%)}\\ \hline
\endfirsthead
\multicolumn{2}{c}%
{\tablename\ \thetable\ -- \textit{Continuação da página anterior}} \\
\hline
\textbf{DATA} & \textbf{Ordem} & \textbf{R\$} & \textbf{Ganho (\%)}\\ \hline
\endhead
\hline \multicolumn{2}{c}{\textit{Continuaçao na próxima página}} \\
\endfoot
\hline
\endlastfoot
	
		3/4/2012	&Sell	&0.22	&\\ \hline
		3/8/2012	&Buy	&0.16	&\\ \hline
		3/10/2012	&Sell	&0.13	&-18.75\%\\ \hline
		3/6/2013	&Buy	&0.12	&\\ \hline
		4/10/2013	&Sell	&0.12	&0.00\%\\ \hline
		3/1/2014	&Buy	&0.11   &	\\ \hline
		5/1/2014	&Sell	&0.11	&0.00\%\\ \hline
		3/5/2014	&Buy	&0.1	&\\ \hline
		2/6/2014	&Sell	&0.09	&-10.00\%\\ \hline
		4/6/2014	&Buy	&0.1	&\\ \hline
		4/6/2014	&Sell	&0.1	&0.00\%\\ \hline
		3/8/2014	&Buy	&0.09	&\\ \hline
		5/8/2014	&Sell	&0.09	&0.00\%\\ \hline

	{} 		&{}		&\textbf{\% Acumulado} 	&\textbf{-28,75\%}

\label{t1}
\end{longtable}
\end{center}


\begin{center}
\begin{longtable}{| p{4cm} | p{4cm} |p{4cm} |p{2cm} |}
\caption*{Agente A2: Ação RPMG3.SA} \\
\hline
\textbf{DATA} & \textbf{Ordem} & \textbf{R\$} & \textbf{Ganho (\%)}\\ \hline
\endfirsthead
\multicolumn{2}{c}%
{\tablename\ \thetable\ -- \textit{Continuação da página anterior}} \\
\hline
\textbf{DATA} & \textbf{Ordem} & \textbf{R\$} & \textbf{Ganho (\%)}\\ \hline
\endhead
\hline \multicolumn{2}{c}{\textit{Continuaçao na próxima página}} \\
\endfoot
\hline
\endlastfoot

	1/4/2012	&Buy	&0.11	&\\ \hline
	2/4/2012	&Sell	&0.1	&-9.09\%\\ \hline
	4/3/2013	&Buy	&0.03	&\\ \hline
	2/4/2013	&Sell	&0.03	&0.00\%\\ \hline
	2/5/2013	&Buy	&0.03	&\\ \hline
	2/10/2013	&Sell	&0.03	&0.00\%\\ \hline
	3/11/2014	&Buy	&0.33	&\\ \hline
	2/2/2015	&Sell	&0.25	&-24.24\%\\ \hline

	{} 		&{}		&\textbf{\% Acumulado} 	&\textbf{-33,33\%}

\label{t1}
\end{longtable}
\end{center}

\begin{center}
\begin{longtable}{| p{4cm} | p{4cm} |p{4cm} |p{2cm} |}
\caption*{Agente A2: Ação SCLO4.SA} \\
\hline
\textbf{DATA} & \textbf{Ordem} & \textbf{R\$} & \textbf{Ganho (\%)}\\ \hline
\endfirsthead
\multicolumn{2}{c}%
{\tablename\ \thetable\ -- \textit{Continuação da página anterior}} \\
\hline
\textbf{DATA} & \textbf{Ordem} & \textbf{R\$} & \textbf{Ganho (\%)}\\ \hline
\endhead
\hline \multicolumn{2}{c}{\textit{Continuaçao na próxima página}} \\
\endfoot
\hline
\endlastfoot

	5/1/2012	&Sell	&3.79	&\\ \hline
	1/1/2012	&Buy	&3.89	&\\ \hline
	3/1/2012	&Sell	&3.7	&-4.88\%\\ \hline
	4/3/2012	&Buy	&3.85	&\\ \hline
	2/3/2012	&Sell	&3.85	&0.00\%\\ \hline
	2/4/2012	&Buy	&3.85	&\\ \hline

	{} 		&{}		&\textbf{\% Acumulado} 	&\textbf{-4,88\%}

\label{t1}
\end{longtable}
\end{center}

\begin{center}
\begin{longtable}{| p{4cm} | p{4cm} |p{4cm} |p{2cm} |}
\caption*{Agente A2: Ação TXRX4.SA} \\
\hline
\textbf{DATA} & \textbf{Ordem} & \textbf{R\$} & \textbf{Ganho (\%)}\\ \hline
\endfirsthead
\multicolumn{2}{c}%
{\tablename\ \thetable\ -- \textit{Continuação da página anterior}} \\
\hline
\textbf{DATA} & \textbf{Ordem} & \textbf{R\$} & \textbf{Ganho (\%)}\\ \hline
\endhead
\hline \multicolumn{2}{c}{\textit{Continuaçao na próxima página}} \\
\endfoot
\hline
\endlastfoot

	
	4/2/2012	&Buy	&0.44	&\\ \hline
	1/9/2012	&Sell	&0.27	&-38.64\%\\ \hline
	5/6/2014	&Buy	&0.28	&\\ \hline
	4/0/2015	&Sell	&0.22	&-21.43\%\\ \hline
	4/2/2015	&Buy	&0.28	&\\ \hline
	{} 		&{}		&\textbf{\% Acumulado} 	&\textbf{-60,07\%}

\label{t1}
\end{longtable}
\end{center}

\begin{center}
\begin{longtable}{| p{4cm} | p{4cm} |p{4cm} |p{2cm} |}
\caption*{Agente A3: Ação MGEL4.SA} \\
\hline
\textbf{DATA} & \textbf{Ordem} & \textbf{R\$} & \textbf{Ganho (\%)}\\ \hline
\endfirsthead
\multicolumn{2}{c}%
{\tablename\ \thetable\ -- \textit{Continuação da página anterior}} \\
\hline
\textbf{DATA} & \textbf{Ordem} & \textbf{R\$} & \textbf{Ganho (\%)}\\ \hline
\endhead
\hline \multicolumn{2}{c}{\textit{Continuaçao na próxima página}} \\
\endfoot
\hline
\endlastfoot

	2/8/2013	&Sell	&0.22	&\\ \hline
	4/9/2013	&Buy	&0.17	&\\ \hline
	5/2/2014	&Sell	&0.12	&-29.41\%\\ \hline
	2/2/2014	&Buy	&0.11	&\\ \hline
	1/2/2014	&Sell	&0.12	&9.09\%\\ \hline
	4/3/2014	&Buy	&0.11	&\\ \hline
	5/5/2014	&Sell	&0.2	&81.82\%\\ \hline
	5/9/2014	&Buy	&0.15	&\\ \hline
	3/1/2015	&Sell	&0.14	&-6.67\%\\ \hline
		
	{} 		&{}		&\textbf{\% Acumulado} 	&\textbf{54,83\%}

\label{t1}
\end{longtable}
\end{center}

\begin{center}
\begin{longtable}{| p{4cm} | p{4cm} |p{4cm} |p{2cm} |}
\caption*{Agente A3: Ação SPRI3.SA} \\
\hline
\textbf{DATA} & \textbf{Ordem} & \textbf{R\$} & \textbf{Ganho (\%)}\\ \hline
\endfirsthead
\multicolumn{2}{c}%
{\tablename\ \thetable\ -- \textit{Continuação da página anterior}} \\
\hline
\textbf{DATA} & \textbf{Ordem} & \textbf{R\$} & \textbf{Ganho (\%)}\\ \hline
\endhead
\hline \multicolumn{2}{c}{\textit{Continuaçao na próxima página}} \\
\endfoot
\hline
\endlastfoot

	3/10/2012	&Sell	&0.44	&\\ \hline
	3/2/2013	&Buy	&0.49	&\\ \hline
	1/8/2013	&Sell	&0.42	&-14.29\%\\ \hline
	2/9/2013	&Buy	&0.36	&\\ \hline
	1/0/2014	&Sell	&0.39	&8.33\%\\ \hline
	3/2/2014	&Buy	&0.37	&\\ \hline
		
	{} 		&{}		&\textbf{\% Acumulado} 	&\textbf{-5,96\%}

\label{t1}
\end{longtable}
\end{center}

\begin{center}
\begin{longtable}{| p{4cm} | p{4cm} |p{4cm} |p{2cm} |}
\caption*{Agente A3: Ação TCNO3.SA} \\
\hline
\textbf{DATA} & \textbf{Ordem} & \textbf{R\$} & \textbf{Ganho (\%)}\\ \hline
\endfirsthead
\multicolumn{2}{c}%
{\tablename\ \thetable\ -- \textit{Continuação da página anterior}} \\
\hline
\textbf{DATA} & \textbf{Ordem} & \textbf{R\$} & \textbf{Ganho (\%)}\\ \hline
\endhead
\hline \multicolumn{2}{c}{\textit{Continuaçao na próxima página}} \\
\endfoot
\hline
\endlastfoot

	1/6/2013	&Sell	&0.16	&\\ \hline
	5/10/2013	&Buy	&0.08	&\\ \hline
	2/0/2014	&Sell	&0.09	&12.50\%\\ \hline
	4/2/2014	&Buy	&0.08	&	\\ \hline
	4/3/2014	&Sell	&0.13	&62.50\%\\ \hline
	4/5/2014	&Buy	&0.1	&\\ \hline
	1/5/2014	&Sell	&0.11	&10.00\%\\ \hline
	4/9/2014	&Buy	&0.22	&\\ \hline
	2/2/2015	&Sell	&0.14	&-36.36\%\\ \hline		
	{} 		&{}		&\textbf{\% Acumulado} 	&\textbf{48,64\%}

\label{t1}
\end{longtable}
\end{center}

\begin{center}
\begin{longtable}{| p{4cm} | p{4cm} |p{4cm} |p{2cm} |}
\caption*{Agente A4: Ação BPHA3.SA} \\
\hline
\textbf{DATA} & \textbf{Ordem} & \textbf{R\$} & \textbf{Ganho (\%)}\\ \hline
\endfirsthead
\multicolumn{2}{c}%
{\tablename\ \thetable\ -- \textit{Continuação da página anterior}} \\
\hline
\textbf{DATA} & \textbf{Ordem} & \textbf{R\$} & \textbf{Ganho (\%)}\\ \hline
\endhead
\hline \multicolumn{2}{c}{\textit{Continuaçao na próxima página}} \\
\endfoot
\hline
\endlastfoot

	2/4/2012	&Sell	&9.32	&\\ \hline
	3/6/2012	&Buy	&10.93	&\\ \hline
	2/9/2012	&Sell	&12.35	&12.99\%\\ \hline
	1/10/2012	&Buy	&12.69	&\\ \hline
	1/2/2013	&Sell	&13.9	&9.54\%\\ \hline
	2/9/2013	&Buy	&8.55	&\\ \hline
	1/10/2013	&Sell	&7.45	&-12.87\%\\ \hline
	1/5/2014	&Buy	&3.73	&\\ \hline
	1/6/2014	&Sell	&3.3	&-11.53\%\\ \hline
	2/7/2014	&Buy	&3.89	&\\ \hline
	1/9/2014	&Sell	&3.77	&-3.08\%\\ \hline

	{} 		&{}		&\textbf{\% Acumulado} 	&\textbf{4,95\%}

\label{t1}
\end{longtable}
\end{center}


\begin{center}
\begin{longtable}{| p{4cm} | p{4cm} |p{4cm} |p{2cm} |}
\caption*{Agente A4: Ação CBEE3.SA} \\
\hline
\textbf{DATA} & \textbf{Ordem} & \textbf{R\$} & \textbf{Ganho (\%)}\\ \hline
\endfirsthead
\multicolumn{2}{c}%
{\tablename\ \thetable\ -- \textit{Continuação da página anterior}} \\
\hline
\textbf{DATA} & \textbf{Ordem} & \textbf{R\$} & \textbf{Ganho (\%)}\\ \hline
\endhead
\hline \multicolumn{2}{c}{\textit{Continuaçao na próxima página}} \\
\endfoot
\hline
\endlastfoot

	2/1/2012	&Buy	&1.8	&\\ \hline
	4/3/2012	&Sell	&1.31	&-27.22\%\\ \hline
	1/4/2012	&Buy	&1.7	&\\ \hline
	1/5/2012	&Sell	&1.58	&-7.06\%\\ \hline
	1/7/2012	&Buy	&1.46	&\\ \hline
	3/9/2012	&Sell	&1.3	&-10.96\%\\ \hline
	5/4/2013	&Buy	&1.75	&\\ \hline
	4/6/2013	&Sell	&1.16	&-33.71\%\\ \hline
	4/7/2013	&Buy	&1.33	&\\ \hline
	3/9/2013	&Sell	&1.29	&-3.01\%\\ \hline
	2/11/2013	&Buy	&1.29	&\\ \hline
	3/0/2014	&Sell	&1.2	&-6.98\%\\ \hline
	4/3/2014	&Buy	&1.14	&\\ \hline
	2/4/2014	&Sell	&1.2	&5.26\%\\ \hline
	4/4/2014	&Buy	&1.24	&\\ \hline
	5/6/2014	&Sell	&1.25	&0.81\%\\ \hline
	5/7/2014	&Buy	&1.23	&\\ \hline
	1/7/2014	&Sell	&1.2	&-2.44\%\\ \hline
	3/8/2014	&Buy	&1.16	&\\ \hline
	4/8/2014	&Sell	&1.15	&-0.86\%\\ \hline
	1/11/2014	&Buy	&1.05	&\\ \hline
	1/0/2015	&Sell	&1.09	&3.81\%\\ \hline
	3/0/2015	&Buy	&1		&\\ \hline
	1/1/2015	&Sell	&1.09	&9.00\%\\ \hline
	5/1/2015	&Buy	&0.95	&\\ \hline
	{} 		&{}		&\textbf{\% Acumulado} 	&\textbf{-73,36\%}

\label{t1}
\end{longtable}
\end{center}


\begin{center}
\begin{longtable}{| p{4cm} | p{4cm} |p{4cm} |p{2cm} |}
\caption*{Agente A4: Ação HETA4.SA} \\
\hline
\textbf{DATA} & \textbf{Ordem} & \textbf{R\$} & \textbf{Ganho (\%)}\\ \hline
\endfirsthead
\multicolumn{2}{c}%
{\tablename\ \thetable\ -- \textit{Continuação da página anterior}} \\
\hline
\textbf{DATA} & \textbf{Ordem} & \textbf{R\$} & \textbf{Ganho (\%)}\\ \hline
\endhead
\hline \multicolumn{2}{c}{\textit{Continuaçao na próxima página}} \\
\endfoot
\hline
\endlastfoot

	5/2/2012	&Sell	&0.37	&\\ \hline
	1/7/2012	&Buy	&0.25	&\\ \hline
	4/8/2012	&Sell	&0.23	&-8.00\%\\ \hline
	2/0/2013	&Buy	&0.16	&\\ \hline
	4/0/2013	&Sell	&0.17	&6.25\%\\ \hline
	3/4/2013	&Buy	&0.14	&\\ \hline
	1/6/2013	&Sell	&0.28	&100.00\%\\ \hline
	4/8/2013	&Buy	&0.32	&\\ \hline
	1/9/2013	&Sell	&0.27	&-15.63\%\\ \hline
	4/11/2013	&Buy	&0.31	&\\ \hline
	3/1/2014	&Sell	&0.31	&0.00\%\\ \hline
	1/6/2014	&Buy	&0.33	&\\ \hline
	2/9/2014	&Sell	&0.4	&21.21\%\\ \hline
	3/2/2015	&Buy	&0.34	&\\ \hline

	{} 		&{}		&\textbf{\% Acumulado} 	&\textbf{103,83\%}

\label{t1}
\end{longtable}
\end{center}

\begin{center}
\begin{longtable}{| p{4cm} | p{4cm} |p{4cm} |p{2cm} |}
\caption*{Agente A4: Ação MLFT4.SA} \\
\hline
\textbf{DATA} & \textbf{Ordem} & \textbf{R\$} & \textbf{Ganho (\%)}\\ \hline
\endfirsthead
\multicolumn{2}{c}%
{\tablename\ \thetable\ -- \textit{Continuação da página anterior}} \\
\hline
\textbf{DATA} & \textbf{Ordem} & \textbf{R\$} & \textbf{Ganho (\%)}\\ \hline
\endhead
\hline \multicolumn{2}{c}{\textit{Continuaçao na próxima página}} \\
\endfoot
\hline
\endlastfoot

	
	3/5/2012	&Sell	&1.7	&\\ \hline
	1/7/2012	&Buy	&1.57	&\\ \hline
	1/8/2012	&Sell	&1.55	&-1.27\%\\ \hline
	2/9/2012	&Buy	&1.69	&\\ \hline
	4/11/2012	&Sell	&1.57	&-7.10\%\\ \hline
	1/0/2013	&Buy	&1.7	&\\ \hline
	1/3/2013	&Sell	&1.72	&1.18\%\\ \hline
	5/9/2013	&Buy	&1.58	&\\ \hline
	2/10/2013	&Sell	&1.5	&-5.06\%\\ \hline
	2/11/2013	&Buy	&1.6	&\\ \hline
	4/1/2014	&Sell	&1.6	&0.00\%\\ \hline
	3/7/2014	&Buy	&1.4	&\\ \hline
	3/7/2014	&Sell	&1.34	&-4.29\%\\ \hline
	3/9/2014	&Buy	&1.36	&\\ \hline
	2/10/2014	&Sell	&1.22	&-10.29\%\\ \hline

	{} 		&{}		&\textbf{\% Acumulado} 	&\textbf{-28,83\%}

\label{t1}
\end{longtable}
\end{center}

\begin{center}
\begin{longtable}{| p{4cm} | p{4cm} |p{4cm} |p{2cm} |}
\caption*{Agente A4: Ação VVAR3.SA} \\
\hline
\textbf{DATA} & \textbf{Ordem} & \textbf{R\$} & \textbf{Ganho (\%)}\\ \hline
\endfirsthead
\multicolumn{2}{c}%
{\tablename\ \thetable\ -- \textit{Continuação da página anterior}} \\
\hline
\textbf{DATA} & \textbf{Ordem} & \textbf{R\$} & \textbf{Ganho (\%)}\\ \hline
\endhead
\hline \multicolumn{2}{c}{\textit{Continuaçao na próxima página}} \\
\endfoot
\hline
\endlastfoot

	2/3/2012	&Sell	&20	 &\\ \hline
	4/7/2012	&Buy	&19	 &\\ \hline
	1/8/2012	&Sell	&18	 &-5.26\%\\ \hline
	1/11/2012	&Buy	&18	 &\\ \hline
	1/0/2013	&Sell	&18	 &0.00\%\\ \hline
	2/1/2013	&Buy	&18.02	&\\ \hline
	2/6/2013	&Sell	&23		&27.64\%\\ \hline
	2/7/2013	&Buy	&24	   &\\ \hline
	5/10/2013	&Sell	&8.97	&-62.63\%\\ \hline
	5/3/2014	&Buy	&7.1	&\\ \hline
	1/6/2014	&Sell	&7.55	&6.34\%\\ \hline
	4/8/2014	&Buy	&8.09	&\\ \hline
	5/9/2014	&Sell	&7	    &-13.47\%\\ \hline
	4/10/2014	&Buy	&7.5	&\\ \hline
	4/11/2014	&Sell	&5.13	&-31.60\%\\ \hline
	5/0/2015	&Buy	&7.38	&\\ \hline

	{} 		&{}		&\textbf{\% Acumulado} 	&\textbf{-78,98\%}

\label{t1}
\end{longtable}
\end{center}

\begin{center}
\begin{longtable}{| p{4cm} | p{4cm} |p{4cm} |p{2cm} |}
\caption*{Agente A5: Ação JHSF3.SA} \\
\hline
\textbf{DATA} & \textbf{Ordem} & \textbf{R\$} & \textbf{Ganho (\%)}\\ \hline
\endfirsthead
\multicolumn{2}{c}%
{\tablename\ \thetable\ -- \textit{Continuação da página anterior}} \\
\hline
\textbf{DATA} & \textbf{Ordem} & \textbf{R\$} & \textbf{Ganho (\%)}\\ \hline
\endhead
\hline \multicolumn{2}{c}{\textit{Continuaçao na próxima página}} \\
\endfoot
\hline
\endlastfoot

	3/4/2012	&Sell	&5.96	&\\ \hline
	2/7/2012	&Buy	&6.75	&\\ \hline
	1/2/2013	&Sell	&8		&18.52\%\\ \hline
	4/8/2013	&Buy	&6.72	&\\ \hline
	4/9/2013	&Sell	&5.58	&-16.96\%\\ \hline
	4/1/2014	&Buy	&4.1	&\\ \hline
	1/2/2014	&Sell	&4		&-2.44\%\\ \hline
	5/6/2014	&Buy	&4		&\\ \hline
	5/8/2014	&Sell	&3.84	&-4.00\%\\ \hline

	{} 		&{}		&\textbf{\% Acumulado} 	&\textbf{-4,88\%}

\label{t1}
\end{longtable}
\end{center}


\begin{center}
\begin{longtable}{| p{4cm} | p{4cm} |p{4cm} |p{2cm} |}
\caption*{Agente A5: Ação MMXM3.SA} \\
\hline
\textbf{DATA} & \textbf{Ordem} & \textbf{R\$} & \textbf{Ganho (\%)}\\ \hline
\endfirsthead
\multicolumn{2}{c}%
{\tablename\ \thetable\ -- \textit{Continuação da página anterior}} \\
\hline
\textbf{DATA} & \textbf{Ordem} & \textbf{R\$} & \textbf{Ganho (\%)}\\ \hline
\endhead
\hline \multicolumn{2}{c}{\textit{Continuaçao na próxima página}} \\
\endfoot
\hline
\endlastfoot

	5/4/2012	&Sell	&8.18	&\\ \hline
	3/0/2013	&Buy	&3.95	&\\ \hline
	5/1/2013	&Sell	&3.3	&-16.46\%\\ \hline
	3/7/2013	&Buy	&1.99	&\\ \hline
	4/9/2013	&Sell	&1.07	&-46.23\%\\ \hline
	5/0/2014	&Buy	&3.7	&\\ \hline
	2/3/2014	&Sell	&2.54	&-31.35\%\\ \hline
	2/11/2014	&Buy	&0.59	&\\ \hline
	5/2/2015	&Sell	&0.66	&11.86\%\\ \hline

	{} 		&{}		&\textbf{\% Acumulado} 	&\textbf{-82,18\%}

\label{t1}
\end{longtable}
\end{center}


\begin{center}
\begin{longtable}{| p{4cm} | p{4cm} |p{4cm} |p{2cm} |}
\caption*{Agente A5: Ação OGXP3.SA} \\
\hline
\textbf{DATA} & \textbf{Ordem} & \textbf{R\$} & \textbf{Ganho (\%)}\\ \hline
\endfirsthead
\multicolumn{2}{c}%
{\tablename\ \thetable\ -- \textit{Continuação da página anterior}} \\
\hline
\textbf{DATA} & \textbf{Ordem} & \textbf{R\$} & \textbf{Ganho (\%)}\\ \hline
\endhead
\hline \multicolumn{2}{c}{\textit{Continuaçao na próxima página}} \\
\endfoot
\hline
\endlastfoot

	4/4/2012	&Sell	&13.75	&\\ \hline
	1/8/2012	&Buy	&6.63	&\\ \hline
	3/9/2012	&Sell	&4.63	&-30.17\%\\ \hline
	2/0/2014	&Buy	&0.31	&\\ \hline
	5/2/2014	&Sell	&0.25	&-19.35\%\\ \hline
	2/6/2014	&Buy	&0.19	&\\ \hline
	4/7/2014	&Sell	&0.18	&-5.26\%\\ \hline

	{} 		&{}		&\textbf{\% Acumulado} 	&\textbf{-54,78\%}

\label{t1}
\end{longtable}
\end{center}


\begin{center}
\begin{longtable}{| p{4cm} | p{4cm} |p{4cm} |p{2cm} |}
\caption*{Agente A5: Ação VIVR3.SA} \\
\hline
\textbf{DATA} & \textbf{Ordem} & \textbf{R\$} & \textbf{Ganho (\%)}\\ \hline
\endfirsthead
\multicolumn{2}{c}%
{\tablename\ \thetable\ -- \textit{Continuação da página anterior}} \\
\hline
\textbf{DATA} & \textbf{Ordem} & \textbf{R\$} & \textbf{Ganho (\%)}\\ \hline
\endhead
\hline \multicolumn{2}{c}{\textit{Continuaçao na próxima página}} \\
\endfoot
\hline
\endlastfoot

	5/1/2012	&Buy	&2.14	&\\ \hline
	3/4/2012	&Sell	&1.93	&-9.81\%\\ \hline
	4/1/2013	&Buy	&1.04	&\\ \hline
	5/3/2013	&Sell	&0.63	&-39.42\%\\ \hline
	5/9/2013	&Buy	&0.32	&\\ \hline
	4/0/2014	&Sell	&0.25	&-21.88\%\\ \hline
	1/7/2014	&Buy	&0.21	&\\ \hline
	3/9/2014	&Sell	&0.12	&-42.86\%\\ \hline

	{} 		&{}		&\textbf{\% Acumulado} 	&\textbf{-113,97\%}

\label{t1}
\end{longtable}
\end{center}


\begin{center}
\begin{longtable}{| p{4cm} | p{4cm} |p{4cm} |p{2cm} |}
\caption*{Agente A6: Ação CRIV4.SA} \\
\hline
\textbf{DATA} & \textbf{Ordem} & \textbf{R\$} & \textbf{Ganho (\%)}\\ \hline
\endfirsthead
\multicolumn{2}{c}%
{\tablename\ \thetable\ -- \textit{Continuação da página anterior}} \\
\hline
\textbf{DATA} & \textbf{Ordem} & \textbf{R\$} & \textbf{Ganho (\%)}\\ \hline
\endhead
\hline \multicolumn{2}{c}{\textit{Continuaçao na próxima página}} \\
\endfoot
\hline
\endlastfoot

	5/2/2012	&Sell	&4.32	&\\ \hline
	3/5/2012	&Buy	&3.99	&\\ \hline
	5/7/2012	&Sell	&4.05	&1.50\%\\ \hline
	2/5/2013	&Buy	&5.19	&\\ \hline

	{} 		&{}		&\textbf{\% Acumulado} 	&\textbf{1,5\%}

\label{t1}
\end{longtable}
\end{center}

\begin{center}
\begin{longtable}{| p{4cm} | p{4cm} |p{4cm} |p{2cm} |}
\caption*{Agente A6: Ação MWET4.SA} \\
\hline
\textbf{DATA} & \textbf{Ordem} & \textbf{R\$} & \textbf{Ganho (\%)}\\ \hline
\endfirsthead
\multicolumn{2}{c}%
{\tablename\ \thetable\ -- \textit{Continuação da página anterior}} \\
\hline
\textbf{DATA} & \textbf{Ordem} & \textbf{R\$} & \textbf{Ganho (\%)}\\ \hline
\endhead
\hline \multicolumn{2}{c}{\textit{Continuaçao na próxima página}} \\
\endfoot
\hline
\endlastfoot

	
	4/8/2012	&Sell	&1.18	&\\ \hline
	4/10/2012	&Buy	&1.12	&\\ \hline
	2/3/2013	&Sell	&0.94	&-16.07\%\\ \hline
	4/3/2013	&Buy	&0.78	&\\ \hline
	1/3/2013	&Sell	&1.1	&41.03\%\\ \hline
	2/6/2013	&Buy	&0.91	&\\ \hline
	5/7/2013	&Sell	&1.09	&19.78\%\\ \hline
	3/8/2013	&Buy	&0.91	&\\ \hline
	2/8/2013	&Sell	&0.97	&6.59\%\\ \hline
	4/1/2014	&Buy	&1.35	&\\ \hline
	2/1/2014	&Sell	&1.6	&18.52\%\\ \hline
	1/1/2014	&Buy	&1.46	&\\ \hline
	5/1/2014	&Sell	&1.56	&6.85\%\\ \hline
	2/2/2014	&Buy	&1.38	&\\ \hline
	5/2/2014	&Sell	&1.63	&18.12\%\\ \hline
	5/3/2014	&Buy	&1.47	&\\ \hline
	{} 		&{}		&\textbf{\% Acumulado} 	&\textbf{94,82\%}

\label{t1}
\end{longtable}
\end{center}


\begin{center}
\begin{longtable}{| p{4cm} | p{4cm} |p{4cm} |p{2cm} |}
\caption*{Agente A6: Ação RJCP3.SA} \\
\hline
\textbf{DATA} & \textbf{Ordem} & \textbf{R\$} & \textbf{Ganho (\%)}\\ \hline
\endfirsthead
\multicolumn{2}{c}%
{\tablename\ \thetable\ -- \textit{Continuação da página anterior}} \\
\hline
\textbf{DATA} & \textbf{Ordem} & \textbf{R\$} & \textbf{Ganho (\%)}\\ \hline
\endhead
\hline \multicolumn{2}{c}{\textit{Continuaçao na próxima página}} \\
\endfoot
\hline
\endlastfoot

	3/9/2013	&Sell	&0.02	&\\ \hline
	1/11/2013	&Buy	&0.01	&\\ \hline

	{} 		&{}		&\textbf{\% Acumulado} 	&\textbf{0,0\%}

\label{t1}
\end{longtable}
\end{center}


\begin{center}
\begin{longtable}{| p{4cm} | p{4cm} |p{4cm} |p{2cm} |}
\caption*{Agente A6: Ação SNSY5.SA} \\
\hline
\textbf{DATA} & \textbf{Ordem} & \textbf{R\$} & \textbf{Ganho (\%)}\\ \hline
\endfirsthead
\multicolumn{2}{c}%
{\tablename\ \thetable\ -- \textit{Continuação da página anterior}} \\
\hline
\textbf{DATA} & \textbf{Ordem} & \textbf{R\$} & \textbf{Ganho (\%)}\\ \hline
\endhead
\hline \multicolumn{2}{c}{\textit{Continuaçao na próxima página}} \\
\endfoot
\hline
\endlastfoot

	1/1/2012	&Sell	&4.06	&\\ \hline
	2/4/2012	&Buy	&3.66	&\\ \hline
	5/7/2012	&Sell	&3.26	&-10.93\%\\ \hline
	4/9/2012	&Buy	&2.81	&\\ \hline
	2/9/2013	&Sell	&1.26	&-55.16\%\\ \hline
	4/10/2013	&Buy	&0.95	&\\ \hline
	4/3/2014	&Sell	&0.87	&-8.42\%\\ \hline
	3/4/2014	&Buy	&0.8	&\\ \hline
	5/7/2014	&Sell	&0.89	&11.25\%\\ \hline
	2/7/2014	&Buy	&0.81	&\\ \hline

	{} 		&{}		&\textbf{\% Acumulado} 	&\textbf{-63,26\%}

\label{t1}
\end{longtable}
\end{center}


\begin{center}
\begin{longtable}{| p{4cm} | p{4cm} |p{4cm} |p{2cm} |}
\caption*{Agente A7: Ação CBMA4.SA} \\
\hline
\textbf{DATA} & \textbf{Ordem} & \textbf{R\$} & \textbf{Ganho (\%)}\\ \hline
\endfirsthead
\multicolumn{2}{c}%
{\tablename\ \thetable\ -- \textit{Continuação da página anterior}} \\
\hline
\textbf{DATA} & \textbf{Ordem} & \textbf{R\$} & \textbf{Ganho (\%)}\\ \hline
\endhead
\hline \multicolumn{2}{c}{\textit{Continuaçao na próxima página}} \\
\endfoot
\hline
\endlastfoot

	
	5/2/2012	&Sell	&0.08	&\\ \hline
	2/7/2012	&Buy	&0.07	&\\ \hline
	5/8/2012	&Sell	&0.05	&-28.57\%\\ \hline
	1/9/2012	&Buy	&0.07	&\\ \hline
	5/11/2012	&Sell	&0.07	&0.00\%\\ \hline
	3/1/2013	&Buy	&0.06	&\\ \hline
	5/2/2013	&Sell	&0.05	&-16.67\%\\ \hline
	1/4/2013	&Buy	&0.05	&\\ \hline
	3/5/2013	&Sell	&0.05	&0.00\%\\ \hline
	3/6/2013	&Buy	&0.05	&\\ \hline
	1/8/2013	&Sell	&0.04	&-20.00\%\\ \hline
	3/10/2013	&Buy	&0.04	&\\ \hline
	4/0/2014	&Sell	&0.05	&25.00\%\\ \hline
	1/1/2014	&Buy	&0.04	&\\ \hline
	2/2/2014	&Sell	&0.04	&0.00\%\\ \hline
	5/6/2014	&Buy	&0.05	&\\ \hline
	4/9/2014	&Sell	&0.05	&0.00\%\\ \hline

	{} 		&{}		&\textbf{\% Acumulado} 	&\textbf{-40,24\%}

\label{t1}
\end{longtable}
\end{center}


\begin{center}
\begin{longtable}{| p{4cm} | p{4cm} |p{4cm} |p{2cm} |}
\caption*{Agente A7: Ação DTCY3.SA} \\
\hline
\textbf{DATA} & \textbf{Ordem} & \textbf{R\$} & \textbf{Ganho (\%)}\\ \hline
\endfirsthead
\multicolumn{2}{c}%
{\tablename\ \thetable\ -- \textit{Continuação da página anterior}} \\
\hline
\textbf{DATA} & \textbf{Ordem} & \textbf{R\$} & \textbf{Ganho (\%)}\\ \hline
\endhead
\hline \multicolumn{2}{c}{\textit{Continuaçao na próxima página}} \\
\endfoot
\hline
\endlastfoot

	5/2/2012	&Sell	&0.91	&\\ \hline
	4/4/2012	&Buy	&0.9	&\\ \hline
	3/4/2012	&Sell	&0.8	&-11.11\%\\ \hline
	1/8/2012	&Buy	&0.66	&\\ \hline
	2/9/2012	&Sell	&0.57	&-13.64\%\\ \hline
	2/1/2013	&Buy	&0.66	&\\ \hline
	2/3/2013	&Sell	&0.59	&-10.61\%\\ \hline
	4/3/2013	&Buy	&0.59	&\\ \hline
	3/4/2013	&Sell	&0.51	&-13.56\%\\ \hline
	1/7/2013	&Buy	&0.43	&\\ \hline
	1/8/2013	&Sell	&0.4	&-6.98\%\\ \hline
	2/0/2014	&Buy	&0.4	&\\ \hline
	3/2/2014	&Sell	&0.31	&-22.50\%\\ \hline
	3/4/2014	&Buy	&0.4	&\\ \hline
	4/7/2014	&Sell	&0.45	&12.50\%\\ \hline
	4/8/2014	&Buy	&0.55	&\\ \hline
	3/8/2014	&Sell	&0.51	&-7.27\%\\ \hline
	2/11/2014	&Buy	&0.39	&\\ \hline
	1/11/2014	&Sell	&0.38	&-2.56\%\\ \hline

	{} 		&{}		&\textbf{\% Acumulado} 	&\textbf{-75,73\%}

\label{t1}
\end{longtable}
\end{center}



\begin{center}
\begin{longtable}{| p{4cm} | p{4cm} |p{4cm} |p{2cm} |}
\caption*{Agente A7: Ação GPCP3.SA} \\
\hline
\textbf{DATA} & \textbf{Ordem} & \textbf{R\$} & \textbf{Ganho (\%)}\\ \hline
\endfirsthead
\multicolumn{2}{c}%
{\tablename\ \thetable\ -- \textit{Continuação da página anterior}} \\
\hline
\textbf{DATA} & \textbf{Ordem} & \textbf{R\$} & \textbf{Ganho (\%)}\\ \hline
\endhead
\hline \multicolumn{2}{c}{\textit{Continuaçao na próxima página}} \\
\endfoot
\hline
\endlastfoot

	
	1/2/2012	&Sell	&0.43	&\\ \hline
	3/6/2012	&Buy	&0.37	&\\ \hline
	3/7/2012	&Sell	&0.34	&-8.11\%\\ \hline
	5/0/2013	&Buy	&0.16	&\\ \hline
	4/1/2013	&Sell	&0.16	&0.00\%\\ \hline
	2/1/2013	&Buy	&0.15	&\\ \hline
	1/1/2013	&Sell	&0.15	&0.00\%\\ \hline
	3/1/2013	&Buy	&0.15	&\\ \hline
	5/2/2013	&Sell	&0.15	&0.00\%\\ \hline
	4/5/2013	&Buy	&0.09	&\\ \hline
	3/10/2013	&Sell	&0.41	&355.56\%\\ \hline
	1/11/2013	&Buy	&0.45	&\\ \hline
	2/0/2014	&Sell	&0.5	&11.11\%\\ \hline
	5/1/2014	&Buy	&0.49	&\\ \hline
	4/2/2014	&Sell	&0.47	&-4.08\%\\ \hline
	4/2/2014	&Buy	&0.54	&\\ \hline

	{} 		&{}		&\textbf{\% Acumulado} 	&\textbf{354,48\%}

\label{t1}
\end{longtable}
\end{center}


\begin{center}
\begin{longtable}{| p{4cm} | p{4cm} |p{4cm} |p{2cm} |}
\caption*{Agente A7: Ação PRML3.SA} \\
\hline
\textbf{DATA} & \textbf{Ordem} & \textbf{R\$} & \textbf{Ganho (\%)}\\ \hline
\endfirsthead
\multicolumn{2}{c}%
{\tablename\ \thetable\ -- \textit{Continuação da página anterior}} \\
\hline
\textbf{DATA} & \textbf{Ordem} & \textbf{R\$} & \textbf{Ganho (\%)}\\ \hline
\endhead
\hline \multicolumn{2}{c}{\textit{Continuaçao na próxima página}} \\
\endfoot
\hline
\endlastfoot

	
	1/3/2012	&Sell	&3.39	&\\ \hline
	3/6/2012	&Buy	&2.41	&\\ \hline
	5/8/2012	&Sell	&2.88	&19.50\%\\ \hline
	1/11/2012	&Buy	&2.4	&\\ \hline
	1/1/2013	&Sell	&2.02	&-15.83\%\\ \hline
	1/2/2013	&Buy	&2.02	&\\ \hline
	4/3/2013	&Sell	&1.87	&-7.43\%\\ \hline
	5/7/2013	&Buy	&1.6	&\\ \hline
	1/9/2013	&Sell	&1.25	&-21.88\%\\ \hline
	2/0/2014	&Buy	&1.09	&\\ \hline
	3/0/2014	&Sell	&0.96	&-11.93\%\\ \hline
	1/3/2014	&Buy	&1.22	&\\ \hline
	3/5/2014	&Sell	&1.1	&-9.84\%\\ \hline
	2/7/2014	&Buy	&1.1	&\\ \hline
	2/8/2014	&Sell	&1.02	&-7.27\%\\ \hline

	{} 		&{}		&\textbf{\% Acumulado} 	&\textbf{-54,68\%}

\label{t1}
\end{longtable}
\end{center}


\begin{center}
\begin{longtable}{| p{4cm} | p{4cm} |p{4cm} |p{2cm} |}
\caption*{Agente A7: Ação RCSL4.SA} \\
\hline
\textbf{DATA} & \textbf{Ordem} & \textbf{R\$} & \textbf{Ganho (\%)}\\ \hline
\endfirsthead
\multicolumn{2}{c}%
{\tablename\ \thetable\ -- \textit{Continuação da página anterior}} \\
\hline
\textbf{DATA} & \textbf{Ordem} & \textbf{R\$} & \textbf{Ganho (\%)}\\ \hline
\endhead
\hline \multicolumn{2}{c}{\textit{Continuaçao na próxima página}} \\
\endfoot
\hline
\endlastfoot

	
	5/2/2012	&Sell	&0.23	&\\ \hline
	2/8/2012	&Buy	&0.1	&\\ \hline
	4/8/2012	&Sell	&0.08	&-20.00\%\\ \hline
	4/7/2013	&Buy	&0.05	&\\ \hline
	4/8/2013	&Sell	&0.04	&-20.00\%\\ \hline
	1/8/2013	&Buy	&0.05	&\\ \hline
	5/11/2013	&Sell	&0.08	&60.00\%\\ \hline
	5/1/2014	&Buy	&0.08	&\\ \hline
	4/2/2014	&Sell	&0.06	&-25.00\%\\ \hline
	3/4/2014	&Buy	&0.07	&\\ \hline
	3/5/2014	&Sell	&0.06	&-14.29\%\\ \hline
	2/7/2014	&Buy	&0.05	&\\ \hline
	5/8/2014	&Sell	&0.05	&0.00\%\\ \hline
	4/11/2014	&Buy	&0.04	&\\ \hline
	1/11/2014	&Sell	&0.03	&-25.00\%\\ \hline
	4/2/2015	&Buy	&0.04	&\\ \hline

	{} 		&{}		&\textbf{\% Acumulado} 	&\textbf{-44,29\%}

\label{t1}
\end{longtable}
\end{center}

\subsection{Dados Simulação Perfil Corajoso}


\begin{center}
\begin{longtable}{| p{2cm} | p{10cm} |p{2cm} |}
\caption*{Estratégias Perfil Corajoso e Resultados} \\
\hline
\textbf{Agente} & \textbf{stratégia} & \textbf{Percentual de Ganho} \\ \hline
\endfirsthead
\multicolumn{2}{c}%
{\tablename\ \thetable\ -- \textit{Continuação da página anterior}} \\
\hline
\textbf{Agente} & \textbf{stratégia} & \textbf{Percentual de Ganho} \\ \hline
\endhead
\hline \multicolumn{2}{c}{\textit{Continuaçao na próxima página}} \\
\endfoot
\hline
\endlastfoot

	A1 & Bearish - Bullish & -34,71\% \\ \hline
	A2 & Dark Cloud - Bullish Engulf & -6,45\% \\ \hline
	{} & \textbf{TOTAL GERAL} & -\textbf{18,08\%} 
	
\label{t10}
\end{longtable}
\end{center} 

\begin{center}
\begin{longtable}{| p{4cm} | p{4cm} |p{4cm} |p{2cm} |}
\caption*{Agente A1: Ação BMKS3.SA} \\
\hline
\textbf{DATA} & \textbf{Ordem} & \textbf{R\$} & \textbf{Ganho (\%)}\\ \hline
\endfirsthead
\multicolumn{2}{c}%
{\tablename\ \thetable\ -- \textit{Continuação da página anterior}} \\
\hline
\textbf{DATA} & \textbf{Ordem} & \textbf{R\$} & \textbf{Ganho (\%)}\\ \hline
\endhead
\hline \multicolumn{2}{c}{\textit{Continuaçao na próxima página}} \\
\endfoot
\hline
\endlastfoot

	
	05/02/12	&Sell	&329.9	&\\ \hline
	06/19/12	&Buy	&305.95	&\\ \hline
	06/24/12	&Sell	&295	&-3.58\%\\ \hline
	08/14/12	&Buy	&305	&\\ \hline
	08/28/12	&Sell	&298	&-2.30\%\\ \hline
	09/05/12	&Sell	&290	&\\ \hline
	03/19/13	&Sell	&362	&\\ \hline
	04/01/13	&Buy	&375	&\\ \hline
	05/02/13	&Sell	&351	&-6.40\%\\ \hline
	09/02/13	&Buy	&330	&\\ \hline
	09/03/13	&Sell	&320.06	&-3.012\%\\ \hline
	09/23/13	&Buy	&340	&\\ \hline
	11/20/13	&Sell	&331	&-2.647\%\\ \hline
	01/14/14	&Buy	&340	&\\ \hline
	03/07/14	&Sell	&320.5	&-5.735\%\\ \hline
	03/14/14	&Buy	&350	&\\ \hline
	03/25/14	&Sell	&337	&-3.714\%\\ \hline
	03/26/14	&Buy	&350	&\\ \hline
	03/27/14	&Sell	&330	&-5.714\%\\ \hline
	04/04/14	&Buy	&350	&\\ \hline
	04/18/14	&Sell	&344	&\\ \hline
	04/24/14	&Buy	&349	&\\ \hline
	04/28/14	&Sell	&330	&-5.444\%\\ \hline
	07/24/14	&Buy	&310	&\\ \hline
	08/04/14	&Sell	&299	&-3.548\%\ \\ \hline
	08/13/14	&Buy	&305	&\\ \hline
	09/10/14	&Sell	&300	&-1.639\%\\ \hline
	09/23/14	&Buy	&320	&\\ \hline
	10/02/14	&Sell	&285	&-10.938\%\\ \hline
	11/28/14	&Buy	&298	&\\ \hline
	12/10/14	&Sell	&270	&-9.396\%\\ \hline
	01/16/14	&Buy	&288.98	&\\ \hline
	01/29/14	&Sell	&260.25	&-9.942\%\\ \hline
	02/19/15	&Buy	&275	&\\ \hline

	{} 		&{}		&\textbf{\% Acumulado} 	&\textbf{-54,68\%}

\label{t1}
\end{longtable}
\end{center}


\begin{center}
\begin{longtable}{| p{4cm} | p{4cm} |p{4cm} |p{2cm} |}
\caption*{Agente A1: Ação DAGB11.SA} \\
\hline
\textbf{DATA} & \textbf{Ordem} & \textbf{R\$} & \textbf{Ganho (\%)}\\ \hline
\endfirsthead
\multicolumn{2}{c}%
{\tablename\ \thetable\ -- \textit{Continuação da página anterior}} \\
\hline
\textbf{DATA} & \textbf{Ordem} & \textbf{R\$} & \textbf{Ganho (\%)}\\ \hline
\endhead
\hline \multicolumn{2}{c}{\textit{Continuaçao na próxima página}} \\
\endfoot
\hline
\endlastfoot

	
	02/02/12	&Sell	&204	&\\ \hline
	03/22/12	&Buy	&232	&\\ \hline
	04/19/12	&Buy	&240	&\\ \hline
	05/24/12	&Sell	&237	&-1.25\%\\ \hline
	06/15/12	&Buy	&242	&\\ \hline
	08/09/12	&Sell	&243	&0.413\%\\ \hline
	08/14/12	&Buy	&251	&\\ \hline
	09/28/12	&Sell	&242.5	&-3.386\%\\ \hline
	12/14/12	&Sell	&267	&\\ \hline
	12/17/12	&Buy	&274	&\\ \hline
	01/01/12	&Sell	&272	&-0.730\%\\ \hline
	01/03/12	&Sell	&268	&\\ \hline
	02/12/13	&Sell	&281.78	&\\ \hline
	04/23/13	&Buy	&269.5	&\\ \hline
	06/11/13	&Buy	&288	&\\ \hline
	10/10/13	&Sell	&322	&11.806\%\\ \hline
	01/20/14	&Sell	&400.5	&\\ \hline
	02/20/14	&Buy	&395.99	&\\ \hline
	02/24/14	&Sell	&387	&-2.270\%\\ \hline
	03/03/14	&Buy	&396	&\\ \hline
	06/02/14	&Buy	&375	&	\\ \hline

	{} 		&{}		&\textbf{\% Acumulado} 	&\textbf{4,58\%}

\label{t1}
\end{longtable}
\end{center}



\begin{center}
\begin{longtable}{| p{4cm} | p{4cm} |p{4cm} |p{2cm} |}
\caption*{Agente A2: Ação KLBN4.SA} \\
\hline
\textbf{DATA} & \textbf{Ordem} & \textbf{R\$} & \textbf{Ganho (\%)}\\ \hline
\endfirsthead
\multicolumn{2}{c}%
{\tablename\ \thetable\ -- \textit{Continuação da página anterior}} \\
\hline
\textbf{DATA} & \textbf{Ordem} & \textbf{R\$} & \textbf{Ganho (\%)}\\ \hline
\endhead
\hline \multicolumn{2}{c}{\textit{Continuaçao na próxima página}} \\
\endfoot
\hline
\endlastfoot

	
	03/13/12	&Buy	&9.63	&\\ \hline
	09/21/12	&Sell	&10.07	&4.569\%\\ \hline
	09/25/12	&Sell	&10.01	&\\ \hline
	09/27/12	&Sell	&10.3	&\\ \hline
	11/16/12	&Buy	&11.85	&\\ \hline
	11/21/12	&Buy	&12.14	&\\ \hline
	11/23/12	&Buy	&12	&\\ \hline
	11/26/12	&Buy	&11.8   &	\\ \hline
	11/27/12	&Buy	&11.7	&\\ \hline
	11/28/12	&Buy	&12		&\\ \hline
	11/29/12	&Buy	&12		&\\ \hline
	11/30/12	&Buy	&12.04	&\\ \hline
	04/05/13	&Buy	&14.29	&\\ \hline
	04/08/13	&Buy	&14.42	&\\ \hline
	04/09/13	&Buy	&14.41	&\\ \hline
	04/10/13	&Buy	&14.53	&\\ \hline
	04/12/13	&Buy	&13.76	&\\ \hline
	04/15/13	&Buy	&13.53	&\\ \hline
	04/16/13	&Buy	&13.76	&\\ \hline
	04/17/13	&Buy	&13.34	&\\ \hline
	04/18/13	&Buy	&13.43	&\\ \hline

	{} 		&{}		&\textbf{\% Acumulado} 	&\textbf{4,00\%}

\label{t1}
\end{longtable}
\end{center}


\begin{center}
\begin{longtable}{| p{4cm} | p{4cm} |p{4cm} |p{2cm} |}
\caption*{Agente A2: Ação USIM3.SA} \\
\hline
\textbf{DATA} & \textbf{Ordem} & \textbf{R\$} & \textbf{Ganho (\%)}\\ \hline
\endfirsthead
\multicolumn{2}{c}%
{\tablename\ \thetable\ -- \textit{Continuação da página anterior}} \\
\hline
\textbf{DATA} & \textbf{Ordem} & \textbf{R\$} & \textbf{Ganho (\%)}\\ \hline
\endhead
\hline \multicolumn{2}{c}{\textit{Continuaçao na próxima página}} \\
\endfoot
\hline
\endlastfoot

	09/19/12	&Buy	&13.49	&\\ \hline
	09/20/12	&Buy	&13.29	&\\ \hline
	09/24/12	&Buy	&13.2	&\\ \hline
	09/25/12	&Buy	&11.7	&\\ \hline
	09/26/12	&Buy	&12.22	&\\ \hline
	09/28/12	&Buy	&11.62	&\\ \hline
	10/01/12	&Buy	&11.74	&\\ \hline
	11/28/12	&Buy	&12.66	&\\ \hline
	11/29/12	&Buy	&13.6	&\\ \hline
	11/30/12	&Buy	&13.65	&\\ \hline
	11/03/12	&Buy	&14.06	&\\ \hline
	12/04/12	&Buy	&13.85	&\\ \hline
	11/05/12	&Buy	&13.16	&\\ \hline
	12/06/12	&Buy	&12.92	&\\ \hline
	12/07/12	&Buy	&13.6	&\\ \hline
	12/10/12	&Buy	&13.67	&\\ \hline
	12/11/12	&Buy	&13.52	&\\ \hline
	02/06/12	&Sell	&11.2	-16.976\%\\ \hline
	02/08/13	&Sell	&11.33	&\\ \hline
	02/11/13	&Sell	&11.33	&\\ \hline
	07/31/13	&Sell	&8.77	&\\ \hline
	08/01/13	&Sell	&8.76	&\\ \hline
	08/02/13	&Sell	&8.6	&\\ \hline
	08/05/13	&Sell	&8.77	&\\ \hline
	08/06/13	&Sell	&8.5	&\\ \hline
	08/07/13	&Sell	&8.67	&\\ \hline
	08/08/13	&Sell	&9.09	&\\ \hline
	08/09/13	&Sell	&9.58	&\\ \hline
	08/12/13	&Sell	&9.58	&\\ \hline
	01/03/14	&Buy	&12.35	&\\ \hline
	01/06/14	&Buy	&12.49	&\\ \hline
	01/07/14	&Buy	&12.54	&\\ \hline
	01/08/14	&Buy	&12.39	&\\ \hline
	01/09/14	&Buy	&12.21	&\\ \hline
		

	{} 		&{}		&\textbf{\% Acumulado} 	&\textbf{-16,9\%}

\label{t1}
\end{longtable}
\end{center}


\section{AÇÕES TOMADAS} 

\section{CONCLUSÃO} 


\end{apendicesenv}
