\begin{resumo}[Resumo]
 


 \vspace{\onelineskip}
    Segundo uma pesquisa realizada pelo Instituto Rosenfield em 2012, a pedido da BM\&FBovespa, apenas 1\% da população investe em bolsa de valores. Dentre os motivos elicitados para os que não investem, destacam-se: (i) a falta de conhecimento em como investir, e (ii) a falta de recursos mensais para investir. Diante do exposto, tem-se como objetivo desse TCC foi o desenvolvimento de uma ferramenta que auxilie o cidadão brasileiro a realizar bons investimentos. O objetivo des Trabalho de Conclusão de Curso é contribuir com o contexto financeiro através da construção de uma ferramenta de estratégia financeira apoiada por Sistemas Multiagentes Comportamentais, onde não é exigido do usuário conhecimentos relacionados ao Mercado Financeiro. Assim, a ferramenta desenvolvida abstrai a complexidde de cálculos financeiros comumente utilizados na análise técnica e delega ao usuário somente a decisão de comprar ou vender. Seu desenvolvimento foi conduzido por uma adaptação da metodologia ágil \textit{Scrum} e a implementação foi realizada com a linguagem Java combinada ao framework Jade, adotando ainda boas práticas da Orientação a Objetos visando reduzir o acoplamento e aumentar a coesão do código fonte. Os resultados obtidos foram avaliados de maneira qualitativa através de iterações de pesquisa - ação.

    
 \noindent
 \textbf{Palavras-chaves}: Sistemas Multiagentes, Metodologia de Desenvolvimento Ágil, Mercado Financeiro, Bolsa de Valores.
\end{resumo}
