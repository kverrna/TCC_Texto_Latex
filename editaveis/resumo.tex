\begin{resumo}
 


 \vspace{\onelineskip}
    Segundo uma pesquisa realizada pelo Instituto Rosenfield em 2012, a pedido da BM\&FBovespa, apenas 1\% da população investe em bolsa de valores. Dentre os motivos elicitados para os que não investem, destacam-se: (i) a falta de conhecimento em como investir, e (ii) a falta de recursos mensais para investir. Diante do exposto, tem-se como objetivo desse TCC foi o desenvolvimento de uma ferramenta que auxilie o cidadão brasileiro a realizar bons investimentos. O software a ser desenvolvido está orientado pelo Paradigma de Sistemas Multiagentes. Seu desenvolvimento foi conduzido por uma adaptação da metodologia ágil Scrum e a implementação foi realizada com a linguagem Java combinada ao framework Jade. A intenção é oferecer ao cidadão uma outra alternativa para realizar investimentos em bolsa de valores, de acordo com seu perfil de investidor.

    
 \noindent
 \textbf{Palavras-chaves}: Sistemas Multiagentes, Metodologia de Desenvolvimento Ágil, Mercado Financeiro, Bolsa de Valores.
\end{resumo}
